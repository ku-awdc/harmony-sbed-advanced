% Options for packages loaded elsewhere
\PassOptionsToPackage{unicode}{hyperref}
\PassOptionsToPackage{hyphens}{url}
%
\documentclass[
  ignorenonframetext,
]{beamer}
\usepackage{pgfpages}
\setbeamertemplate{caption}[numbered]
\setbeamertemplate{caption label separator}{: }
\setbeamercolor{caption name}{fg=normal text.fg}
\beamertemplatenavigationsymbolsempty
% Prevent slide breaks in the middle of a paragraph
\widowpenalties 1 10000
\raggedbottom
\setbeamertemplate{part page}{
  \centering
  \begin{beamercolorbox}[sep=16pt,center]{part title}
    \usebeamerfont{part title}\insertpart\par
  \end{beamercolorbox}
}
\setbeamertemplate{section page}{
  \centering
  \begin{beamercolorbox}[sep=12pt,center]{part title}
    \usebeamerfont{section title}\insertsection\par
  \end{beamercolorbox}
}
\setbeamertemplate{subsection page}{
  \centering
  \begin{beamercolorbox}[sep=8pt,center]{part title}
    \usebeamerfont{subsection title}\insertsubsection\par
  \end{beamercolorbox}
}
\AtBeginPart{
  \frame{\partpage}
}
\AtBeginSection{
  \ifbibliography
  \else
    \frame{\sectionpage}
  \fi
}
\AtBeginSubsection{
  \frame{\subsectionpage}
}
\usepackage{amsmath,amssymb}
\usepackage{lmodern}
\usepackage{iftex}
\ifPDFTeX
  \usepackage[T1]{fontenc}
  \usepackage[utf8]{inputenc}
  \usepackage{textcomp} % provide euro and other symbols
\else % if luatex or xetex
  \usepackage{unicode-math}
  \defaultfontfeatures{Scale=MatchLowercase}
  \defaultfontfeatures[\rmfamily]{Ligatures=TeX,Scale=1}
\fi
% Use upquote if available, for straight quotes in verbatim environments
\IfFileExists{upquote.sty}{\usepackage{upquote}}{}
\IfFileExists{microtype.sty}{% use microtype if available
  \usepackage[]{microtype}
  \UseMicrotypeSet[protrusion]{basicmath} % disable protrusion for tt fonts
}{}
\makeatletter
\@ifundefined{KOMAClassName}{% if non-KOMA class
  \IfFileExists{parskip.sty}{%
    \usepackage{parskip}
  }{% else
    \setlength{\parindent}{0pt}
    \setlength{\parskip}{6pt plus 2pt minus 1pt}}
}{% if KOMA class
  \KOMAoptions{parskip=half}}
\makeatother
\usepackage{xcolor}
\newif\ifbibliography
\usepackage{color}
\usepackage{fancyvrb}
\newcommand{\VerbBar}{|}
\newcommand{\VERB}{\Verb[commandchars=\\\{\}]}
\DefineVerbatimEnvironment{Highlighting}{Verbatim}{commandchars=\\\{\}}
% Add ',fontsize=\small' for more characters per line
\usepackage{framed}
\definecolor{shadecolor}{RGB}{248,248,248}
\newenvironment{Shaded}{\begin{snugshade}}{\end{snugshade}}
\newcommand{\AlertTok}[1]{\textcolor[rgb]{0.94,0.16,0.16}{#1}}
\newcommand{\AnnotationTok}[1]{\textcolor[rgb]{0.56,0.35,0.01}{\textbf{\textit{#1}}}}
\newcommand{\AttributeTok}[1]{\textcolor[rgb]{0.77,0.63,0.00}{#1}}
\newcommand{\BaseNTok}[1]{\textcolor[rgb]{0.00,0.00,0.81}{#1}}
\newcommand{\BuiltInTok}[1]{#1}
\newcommand{\CharTok}[1]{\textcolor[rgb]{0.31,0.60,0.02}{#1}}
\newcommand{\CommentTok}[1]{\textcolor[rgb]{0.56,0.35,0.01}{\textit{#1}}}
\newcommand{\CommentVarTok}[1]{\textcolor[rgb]{0.56,0.35,0.01}{\textbf{\textit{#1}}}}
\newcommand{\ConstantTok}[1]{\textcolor[rgb]{0.00,0.00,0.00}{#1}}
\newcommand{\ControlFlowTok}[1]{\textcolor[rgb]{0.13,0.29,0.53}{\textbf{#1}}}
\newcommand{\DataTypeTok}[1]{\textcolor[rgb]{0.13,0.29,0.53}{#1}}
\newcommand{\DecValTok}[1]{\textcolor[rgb]{0.00,0.00,0.81}{#1}}
\newcommand{\DocumentationTok}[1]{\textcolor[rgb]{0.56,0.35,0.01}{\textbf{\textit{#1}}}}
\newcommand{\ErrorTok}[1]{\textcolor[rgb]{0.64,0.00,0.00}{\textbf{#1}}}
\newcommand{\ExtensionTok}[1]{#1}
\newcommand{\FloatTok}[1]{\textcolor[rgb]{0.00,0.00,0.81}{#1}}
\newcommand{\FunctionTok}[1]{\textcolor[rgb]{0.00,0.00,0.00}{#1}}
\newcommand{\ImportTok}[1]{#1}
\newcommand{\InformationTok}[1]{\textcolor[rgb]{0.56,0.35,0.01}{\textbf{\textit{#1}}}}
\newcommand{\KeywordTok}[1]{\textcolor[rgb]{0.13,0.29,0.53}{\textbf{#1}}}
\newcommand{\NormalTok}[1]{#1}
\newcommand{\OperatorTok}[1]{\textcolor[rgb]{0.81,0.36,0.00}{\textbf{#1}}}
\newcommand{\OtherTok}[1]{\textcolor[rgb]{0.56,0.35,0.01}{#1}}
\newcommand{\PreprocessorTok}[1]{\textcolor[rgb]{0.56,0.35,0.01}{\textit{#1}}}
\newcommand{\RegionMarkerTok}[1]{#1}
\newcommand{\SpecialCharTok}[1]{\textcolor[rgb]{0.00,0.00,0.00}{#1}}
\newcommand{\SpecialStringTok}[1]{\textcolor[rgb]{0.31,0.60,0.02}{#1}}
\newcommand{\StringTok}[1]{\textcolor[rgb]{0.31,0.60,0.02}{#1}}
\newcommand{\VariableTok}[1]{\textcolor[rgb]{0.00,0.00,0.00}{#1}}
\newcommand{\VerbatimStringTok}[1]{\textcolor[rgb]{0.31,0.60,0.02}{#1}}
\newcommand{\WarningTok}[1]{\textcolor[rgb]{0.56,0.35,0.01}{\textbf{\textit{#1}}}}
\usepackage{longtable,booktabs,array}
\usepackage{calc} % for calculating minipage widths
\usepackage{caption}
% Make caption package work with longtable
\makeatletter
\def\fnum@table{\tablename~\thetable}
\makeatother
\setlength{\emergencystretch}{3em} % prevent overfull lines
\providecommand{\tightlist}{%
  \setlength{\itemsep}{0pt}\setlength{\parskip}{0pt}}
\setcounter{secnumdepth}{-\maxdimen} % remove section numbering
\ifLuaTeX
  \usepackage{selnolig}  % disable illegal ligatures
\fi
\IfFileExists{bookmark.sty}{\usepackage{bookmark}}{\usepackage{hyperref}}
\IfFileExists{xurl.sty}{\usepackage{xurl}}{} % add URL line breaks if available
\urlstyle{same} % disable monospaced font for URLs
\hypersetup{
  pdftitle={Madrid Advanced training: day 1},
  pdfauthor={Giles Innocent},
  hidelinks,
  pdfcreator={LaTeX via pandoc}}

\title{Madrid Advanced training: day 1}
\author{Giles Innocent}
\date{2023-07-12}

\begin{document}
\frame{\titlepage}

\begin{frame}{Introduction}
\protect\hypertarget{introduction}{}
\Huge``Hello''

\begin{itemize}
\tightlist
\item
  Building on the basic course.
\item
  \textbf{NOT} lecture:practical.
\item
  More intro: practical or discussion.
\item
  You'll get more out of it if you try the exercises \textbf{before}
  looking at the sample code.
\end{itemize}
\end{frame}

\begin{frame}{Simulation}
\protect\hypertarget{simulation}{}
\begin{itemize}
\tightlist
\item
  Why simulate?
\item
  How to simulate

  \begin{itemize}
  \tightlist
  \item
    within JAGS
  \item
    R or equivalent
  \item
    from an identical model to the analysis model
  \item
    from a different model to the analysis
  \end{itemize}
\end{itemize}
\end{frame}

\begin{frame}[fragile]{Simulating in R}
\protect\hypertarget{simulating-in-r}{}
\begin{itemize}
\tightlist
\item
  Functions like rbinom, rpois, rnorm, etc.
\item
  All take a first parameter, n the number of data points you wish to
  simulate
\item
  E.G. Hui-Walter paradigm:
\end{itemize}

\begin{Shaded}
\begin{Highlighting}[]
  \FunctionTok{set.seed}\NormalTok{(}\DecValTok{1}\NormalTok{)}
\NormalTok{  n.sim }\OtherTok{\textless{}{-}} \DecValTok{1}
\NormalTok{  prev }\OtherTok{\textless{}{-}} \FunctionTok{c}\NormalTok{(}\FloatTok{0.25}\NormalTok{, }\FloatTok{0.8}\NormalTok{)}
\NormalTok{  Se}\FloatTok{.1} \OtherTok{\textless{}{-}}\NormalTok{ Se}\FloatTok{.2} \OtherTok{\textless{}{-}} \FloatTok{0.8}
\NormalTok{  Sp}\FloatTok{.1} \OtherTok{\textless{}{-}}\NormalTok{ Sp}\FloatTok{.2} \OtherTok{\textless{}{-}} \FloatTok{0.95}
\NormalTok{  n.sampled }\OtherTok{\textless{}{-}} \FunctionTok{c}\NormalTok{(}\DecValTok{100}\NormalTok{, }\DecValTok{100}\NormalTok{)}
\NormalTok{  test.results }\OtherTok{\textless{}{-}} \FunctionTok{data.frame}\NormalTok{(}\AttributeTok{pp=}\FunctionTok{numeric}\NormalTok{(}\FunctionTok{length}\NormalTok{(prev)),}
                             \AttributeTok{pn=}\FunctionTok{numeric}\NormalTok{(}\FunctionTok{length}\NormalTok{(prev)),}
                             \AttributeTok{np=}\FunctionTok{numeric}\NormalTok{(}\FunctionTok{length}\NormalTok{(prev)),}
                             \AttributeTok{nn=}\FunctionTok{numeric}\NormalTok{(}\FunctionTok{length}\NormalTok{(prev)))}
  \ControlFlowTok{for}\NormalTok{(pop }\ControlFlowTok{in} \DecValTok{1}\SpecialCharTok{:}\FunctionTok{length}\NormalTok{(prev))\{}
\NormalTok{    n.pos }\OtherTok{\textless{}{-}} \FunctionTok{rbinom}\NormalTok{(n.sim,n.sampled[pop],prev[pop])}
\NormalTok{    test.results}\SpecialCharTok{$}\NormalTok{pp[pop] }\OtherTok{\textless{}{-}} \FunctionTok{rbinom}\NormalTok{(n.sim, n.pos, Se}\FloatTok{.1}\SpecialCharTok{*}\NormalTok{Se}\FloatTok{.2}\NormalTok{) }\SpecialCharTok{+} 
      \FunctionTok{rbinom}\NormalTok{(n.sim, n.sampled[pop]}\SpecialCharTok{{-}}\NormalTok{n.pos, (}\DecValTok{1}\SpecialCharTok{{-}}\NormalTok{Sp}\FloatTok{.1}\NormalTok{)}\SpecialCharTok{*}\NormalTok{(}\DecValTok{1}\SpecialCharTok{{-}}\NormalTok{Sp}\FloatTok{.2}\NormalTok{))}
\NormalTok{    test.results}\SpecialCharTok{$}\NormalTok{pn[pop] }\OtherTok{\textless{}{-}} \FunctionTok{rbinom}\NormalTok{(n.sim, n.pos, Se}\FloatTok{.1}\SpecialCharTok{*}\NormalTok{(}\DecValTok{1}\SpecialCharTok{{-}}\NormalTok{Se}\FloatTok{.2}\NormalTok{)) }\SpecialCharTok{+} 
      \FunctionTok{rbinom}\NormalTok{(n.sim, n.sampled[pop]}\SpecialCharTok{{-}}\NormalTok{n.pos, (}\DecValTok{1}\SpecialCharTok{{-}}\NormalTok{Sp}\FloatTok{.1}\NormalTok{)}\SpecialCharTok{*}\NormalTok{Sp}\FloatTok{.2}\NormalTok{)}
\NormalTok{    test.results}\SpecialCharTok{$}\NormalTok{np[pop] }\OtherTok{\textless{}{-}} \FunctionTok{rbinom}\NormalTok{(n.sim, n.pos, (}\DecValTok{1}\SpecialCharTok{{-}}\NormalTok{Se}\FloatTok{.1}\NormalTok{)}\SpecialCharTok{*}\NormalTok{Se}\FloatTok{.2}\NormalTok{) }\SpecialCharTok{+} 
      \FunctionTok{rbinom}\NormalTok{(n.sim, n.sampled[pop]}\SpecialCharTok{{-}}\NormalTok{n.pos, Sp}\FloatTok{.1}\SpecialCharTok{*}\NormalTok{(}\DecValTok{1}\SpecialCharTok{{-}}\NormalTok{Sp}\FloatTok{.2}\NormalTok{))}
\NormalTok{    test.results}\SpecialCharTok{$}\NormalTok{nn[pop] }\OtherTok{\textless{}{-}}\NormalTok{ n.sampled[pop]}\SpecialCharTok{{-}}\NormalTok{test.results}\SpecialCharTok{$}\NormalTok{pp[pop]}\SpecialCharTok{{-}}
\NormalTok{      test.results}\SpecialCharTok{$}\NormalTok{pn[pop] }\SpecialCharTok{{-}}\NormalTok{test.results}\SpecialCharTok{$}\NormalTok{np[pop]}
\NormalTok{  \}}
\end{Highlighting}
\end{Shaded}

\begin{itemize}
\tightlist
\item
  What is wrong with this example?
\end{itemize}

\begin{block}{A better version}
\protect\hypertarget{a-better-version}{}
\begin{Shaded}
\begin{Highlighting}[]
  \FunctionTok{set.seed}\NormalTok{(}\DecValTok{1}\NormalTok{)}
\NormalTok{  n.sim }\OtherTok{\textless{}{-}} \DecValTok{1}
\NormalTok{  prev }\OtherTok{\textless{}{-}} \FunctionTok{c}\NormalTok{(}\FloatTok{0.25}\NormalTok{, }\FloatTok{0.8}\NormalTok{)}
\NormalTok{  Se}\FloatTok{.1} \OtherTok{\textless{}{-}}\NormalTok{ Se}\FloatTok{.2} \OtherTok{\textless{}{-}} \FloatTok{0.8}
\NormalTok{  Sp}\FloatTok{.1} \OtherTok{\textless{}{-}}\NormalTok{ Sp}\FloatTok{.2} \OtherTok{\textless{}{-}} \FloatTok{0.95}
\NormalTok{  cond.prob.pos }\OtherTok{\textless{}{-}} \FunctionTok{c}\NormalTok{(Se}\FloatTok{.1}\SpecialCharTok{*}\NormalTok{Se}\FloatTok{.2}\NormalTok{, }\CommentTok{\# probability a positive individual tests ++ }
\NormalTok{                     (}\DecValTok{1}\SpecialCharTok{{-}}\NormalTok{Se}\FloatTok{.1}\NormalTok{)}\SpecialCharTok{*}\NormalTok{Se}\FloatTok{.2}\NormalTok{, }\CommentTok{\# probability a positive individual tests {-}+ }
\NormalTok{                     Se}\FloatTok{.1}\SpecialCharTok{*}\NormalTok{(}\DecValTok{1}\SpecialCharTok{{-}}\NormalTok{Se}\FloatTok{.2}\NormalTok{), }\CommentTok{\# probability a positive individual tests + }
\NormalTok{                     (}\DecValTok{1}\SpecialCharTok{{-}}\NormalTok{Se}\FloatTok{.1}\NormalTok{)}\SpecialCharTok{*}\NormalTok{(}\DecValTok{1}\SpecialCharTok{{-}}\NormalTok{Se}\FloatTok{.2}\NormalTok{)) }\CommentTok{\# probability a positive individual tests {-}{-}}
\NormalTok{  cond.prob.neg }\OtherTok{\textless{}{-}} \FunctionTok{c}\NormalTok{((}\DecValTok{1}\SpecialCharTok{{-}}\NormalTok{Sp}\FloatTok{.1}\NormalTok{)}\SpecialCharTok{*}\NormalTok{(}\DecValTok{1}\SpecialCharTok{{-}}\NormalTok{Sp}\FloatTok{.2}\NormalTok{), }\CommentTok{\# probability a negative individual tests ++}
\NormalTok{                     Sp}\FloatTok{.1}\SpecialCharTok{*}\NormalTok{(}\DecValTok{1}\SpecialCharTok{{-}}\NormalTok{Sp}\FloatTok{.2}\NormalTok{), }\CommentTok{\# probability a negative individual tests {-}+}
\NormalTok{                     (}\DecValTok{1}\SpecialCharTok{{-}}\NormalTok{Sp}\FloatTok{.1}\NormalTok{)}\SpecialCharTok{*}\NormalTok{Sp}\FloatTok{.2}\NormalTok{, }\CommentTok{\# probability a negative individual tests +{-}}
\NormalTok{                     Sp}\FloatTok{.1}\SpecialCharTok{*}\NormalTok{Sp}\FloatTok{.2}\NormalTok{) }\CommentTok{\# probability a negative individual tests {-}{-}}
\NormalTok{  n.sampled }\OtherTok{\textless{}{-}} \FunctionTok{c}\NormalTok{(}\DecValTok{100}\NormalTok{, }\DecValTok{100}\NormalTok{)}
\NormalTok{  test.results }\OtherTok{\textless{}{-}} \FunctionTok{matrix}\NormalTok{(}\AttributeTok{nrow=}\DecValTok{4}\NormalTok{,}\AttributeTok{ncol=}\FunctionTok{length}\NormalTok{(prev), }\AttributeTok{dimnames=}\FunctionTok{list}\NormalTok{(}\AttributeTok{test.result =} \FunctionTok{c}\NormalTok{(}\StringTok{"pp"}\NormalTok{,}\StringTok{"np"}\NormalTok{,}\StringTok{"pn"}\NormalTok{,}\StringTok{"nn"}\NormalTok{), }\AttributeTok{population =} \FunctionTok{c}\NormalTok{(}\StringTok{"a"}\NormalTok{,}\StringTok{"b"}\NormalTok{)))}
  \ControlFlowTok{for}\NormalTok{(pop }\ControlFlowTok{in} \DecValTok{1}\SpecialCharTok{:}\FunctionTok{length}\NormalTok{(prev))\{}
\NormalTok{    n.pos }\OtherTok{\textless{}{-}} \FunctionTok{rbinom}\NormalTok{(n.sim,n.sampled[pop],prev[pop])}
\NormalTok{    n.neg }\OtherTok{\textless{}{-}}\NormalTok{ n.sampled[pop] }\SpecialCharTok{{-}}\NormalTok{ n.pos}
\NormalTok{    test.results[,pop] }\OtherTok{\textless{}{-}} \FunctionTok{rmultinom}\NormalTok{(n.sim, n.pos, cond.prob.pos) }\SpecialCharTok{+} 
      \FunctionTok{rmultinom}\NormalTok{(n.sim, n.neg, cond.prob.neg)}
\NormalTok{  \}}
\end{Highlighting}
\end{Shaded}
\end{block}
\end{frame}

\begin{frame}{Example}
\protect\hypertarget{example}{}
\begin{itemize}
\tightlist
\item
  3 tests; 1 population
\item
  7 parameters:

  \begin{itemize}
  \tightlist
  \item
    3 test sensitivities
  \item
    3 test specificities
  \item
    1 prevalence
  \end{itemize}
\item
  \(2^3\) combinations: 7 df in the data

  \begin{itemize}
  \tightlist
  \item
    is this identifiable?
  \item
    are the estimates unbiased?
  \item
    what if prevalence is very low \textasciitilde1\%?
  \item
    even with 1000 individuals only \textasciitilde10 are positive
  \item
    can't estimate Se well
  \item
    does a biased estimate of Se bias our estimates of Sp and/or
    prevalence?
  \end{itemize}
\end{itemize}
\end{frame}

\begin{frame}[fragile]{Example R code}
\protect\hypertarget{example-r-code}{}
\begin{Shaded}
\begin{Highlighting}[]
\NormalTok{simulation.}\FloatTok{3.}\NormalTok{test }\OtherTok{\textless{}{-}} \ControlFlowTok{function}\NormalTok{(prev,n.sampled,Se,Sp) \{}
  \ControlFlowTok{if}\NormalTok{((}\FunctionTok{length}\NormalTok{(prev)}\SpecialCharTok{!=}\DecValTok{1}\NormalTok{)}\SpecialCharTok{\&}\NormalTok{(}\FunctionTok{length}\NormalTok{(n.sampled)}\SpecialCharTok{!=}\DecValTok{1}\NormalTok{)}\SpecialCharTok{\&}\NormalTok{(}\FunctionTok{length}\NormalTok{(Se)}\SpecialCharTok{!=}\DecValTok{3}\NormalTok{)}\SpecialCharTok{\&}\NormalTok{(}\FunctionTok{length}\NormalTok{(Sp)}\SpecialCharTok{!=}\DecValTok{3}\NormalTok{)) \{}
    \FunctionTok{print}\NormalTok{(}\StringTok{"Error in parameters sent to simulation.3.test"}\NormalTok{)}
    \FunctionTok{stop}\NormalTok{()}
\NormalTok{  \}}
\NormalTok{  n.pos }\OtherTok{\textless{}{-}} \FunctionTok{rbinom}\NormalTok{(}\DecValTok{1}\NormalTok{,n.sampled,prev)}
\NormalTok{  n.neg }\OtherTok{\textless{}{-}}\NormalTok{ n.sampled }\SpecialCharTok{{-}}\NormalTok{ n.pos}
\NormalTok{  test}\FloatTok{.1} \OtherTok{\textless{}{-}} \FunctionTok{c}\NormalTok{(}\FunctionTok{rbinom}\NormalTok{(n.pos,}\DecValTok{1}\NormalTok{,Se[}\DecValTok{1}\NormalTok{]), }\FunctionTok{rbinom}\NormalTok{(n.neg,}\DecValTok{1}\NormalTok{,(}\DecValTok{1}\SpecialCharTok{{-}}\NormalTok{Sp[}\DecValTok{1}\NormalTok{])))}
\NormalTok{  test}\FloatTok{.2} \OtherTok{\textless{}{-}} \FunctionTok{c}\NormalTok{(}\FunctionTok{rbinom}\NormalTok{(n.pos,}\DecValTok{1}\NormalTok{,Se[}\DecValTok{2}\NormalTok{]), }\FunctionTok{rbinom}\NormalTok{(n.neg,}\DecValTok{1}\NormalTok{,(}\DecValTok{1}\SpecialCharTok{{-}}\NormalTok{Sp[}\DecValTok{2}\NormalTok{])))}
\NormalTok{  test}\FloatTok{.3} \OtherTok{\textless{}{-}} \FunctionTok{c}\NormalTok{(}\FunctionTok{rbinom}\NormalTok{(n.pos,}\DecValTok{1}\NormalTok{,Se[}\DecValTok{3}\NormalTok{]), }\FunctionTok{rbinom}\NormalTok{(n.neg,}\DecValTok{1}\NormalTok{,(}\DecValTok{1}\SpecialCharTok{{-}}\NormalTok{Sp[}\DecValTok{3}\NormalTok{])))}
\NormalTok{  test.results }\OtherTok{\textless{}{-}} \FunctionTok{c}\NormalTok{(}
    \FunctionTok{sum}\NormalTok{(test}\FloatTok{.1} \SpecialCharTok{\&}\NormalTok{ test}\FloatTok{.2} \SpecialCharTok{\&}\NormalTok{ test}\FloatTok{.3}\NormalTok{),}
    \FunctionTok{sum}\NormalTok{(test}\FloatTok{.1} \SpecialCharTok{\&}\NormalTok{ test}\FloatTok{.2} \SpecialCharTok{\&} \SpecialCharTok{!}\NormalTok{test}\FloatTok{.3}\NormalTok{),}
    \FunctionTok{sum}\NormalTok{(test}\FloatTok{.1} \SpecialCharTok{\&} \SpecialCharTok{!}\NormalTok{test}\FloatTok{.2} \SpecialCharTok{\&}\NormalTok{ test}\FloatTok{.3}\NormalTok{),}
    \FunctionTok{sum}\NormalTok{(test}\FloatTok{.1} \SpecialCharTok{\&} \SpecialCharTok{!}\NormalTok{test}\FloatTok{.2} \SpecialCharTok{\&} \SpecialCharTok{!}\NormalTok{test}\FloatTok{.3}\NormalTok{),}
    \FunctionTok{sum}\NormalTok{(}\SpecialCharTok{!}\NormalTok{test}\FloatTok{.1} \SpecialCharTok{\&}\NormalTok{ test}\FloatTok{.2} \SpecialCharTok{\&}\NormalTok{ test}\FloatTok{.3}\NormalTok{),}
    \FunctionTok{sum}\NormalTok{(}\SpecialCharTok{!}\NormalTok{test}\FloatTok{.1} \SpecialCharTok{\&}\NormalTok{ test}\FloatTok{.2} \SpecialCharTok{\&} \SpecialCharTok{!}\NormalTok{test}\FloatTok{.3}\NormalTok{),}
    \FunctionTok{sum}\NormalTok{(}\SpecialCharTok{!}\NormalTok{test}\FloatTok{.1} \SpecialCharTok{\&} \SpecialCharTok{!}\NormalTok{test}\FloatTok{.2} \SpecialCharTok{\&}\NormalTok{ test}\FloatTok{.3}\NormalTok{),}
    \FunctionTok{sum}\NormalTok{(}\SpecialCharTok{!}\NormalTok{test}\FloatTok{.1} \SpecialCharTok{\&} \SpecialCharTok{!}\NormalTok{test}\FloatTok{.2} \SpecialCharTok{\&} \SpecialCharTok{!}\NormalTok{test}\FloatTok{.3}\NormalTok{))}
  \FunctionTok{names}\NormalTok{(test.results) }\OtherTok{\textless{}{-}} \FunctionTok{c}\NormalTok{(}\StringTok{"ppp"}\NormalTok{, }\StringTok{"ppn"}\NormalTok{, }\StringTok{"pnp"}\NormalTok{, }\StringTok{"pnn"}\NormalTok{, }\StringTok{"npp"}\NormalTok{, }\StringTok{"npn"}\NormalTok{, }\StringTok{"nnp"}\NormalTok{, }\StringTok{"nnn"}\NormalTok{)}
  \FunctionTok{return}\NormalTok{(test.results)}
\NormalTok{\}}
\end{Highlighting}
\end{Shaded}
\end{frame}

\begin{frame}[fragile]{Example R/JAGS code}
\protect\hypertarget{example-rjags-code}{}
\begin{Shaded}
\begin{Highlighting}[]
\FunctionTok{cat}\NormalTok{(}
\StringTok{"model\{}
\StringTok{  \# Likelihood part:}
\StringTok{  p.test.result[1] \textless{}{-}prev*Se[1]*Se[2]*Se[3] + (1{-}prev)*(1{-}Sp[1])*(1{-}Sp[2])*(1{-}Sp[3]) \#ppp}
\StringTok{  p.test.result[2] \textless{}{-}prev*Se[1]*Se[2]*(1{-}Se[3]) + (1{-}prev)*(1{-}Sp[1])*(1{-}Sp[2])*Sp[3] \#ppn}
\StringTok{  p.test.result[3] \textless{}{-}prev*Se[1]*(1{-}Se[2])*Se[3] + (1{-}prev)*(1{-}Sp[1])*Sp[2]*(1{-}Sp[3]) \#pnp}
\StringTok{  p.test.result[4] \textless{}{-}prev*Se[1]*(1{-}Se[2])*(1{-}Se[3]) + (1{-}prev)*(1{-}Sp[1])*Sp[2]*Sp[3] \#pnn}
\StringTok{  p.test.result[5] \textless{}{-}prev*(1{-}Se[1])*Se[2]*Se[3] + (1{-}prev)*Sp[1]*(1{-}Sp[2])*(1{-}Sp[3]) \#npp}
\StringTok{  p.test.result[6] \textless{}{-}prev*(1{-}Se[1])*Se[2]*(1{-}Se[3]) + (1{-}prev)*Sp[1]*(1{-}Sp[2])*Sp[3] \#npn}
\StringTok{  p.test.result[7] \textless{}{-}prev*(1{-}Se[1])*(1{-}Se[2])*Se[3] + (1{-}prev)*Sp[1]*Sp[2]*(1{-}Sp[3]) \#nnp}
\StringTok{  p.test.result[8] \textless{}{-}prev*(1{-}Se[1])*(1{-}Se[2])*(1{-}Se[3]) + (1{-}prev)*Sp[1]*Sp[2]*Sp[3] \#nnn}
\StringTok{  test.results \textasciitilde{}dmulti(p.test.result, n.tested)}
\StringTok{    }
\StringTok{  \# Prior part:}
\StringTok{  prev \textasciitilde{} dbeta(1,1)}
\StringTok{  for(test in 1:3)  \{}
\StringTok{    Se[test] \textasciitilde{}dbeta(1,1)}
\StringTok{    Sp[test] \textasciitilde{}dbeta(1,1)}
\StringTok{  \}}
\StringTok{  }
\StringTok{  \# Hooks for automatic integration with R:}
\StringTok{  \#data\# test.results, n.tested}
\StringTok{  \#monitor\# prev, Se, Sp}
\StringTok{\}}
\StringTok{"}\NormalTok{, }\AttributeTok{file =} \StringTok{"three.test.jags"}\NormalTok{)}

\NormalTok{prev}\OtherTok{\textless{}{-}} \FloatTok{0.50}
\NormalTok{n.tested }\OtherTok{\textless{}{-}} \DecValTok{1000}
\NormalTok{Se }\OtherTok{\textless{}{-}} \FunctionTok{c}\NormalTok{(}\FloatTok{0.8}\NormalTok{,}\FloatTok{0.8}\NormalTok{,}\FloatTok{0.95}\NormalTok{)}
\NormalTok{Sp }\OtherTok{\textless{}{-}} \FunctionTok{c}\NormalTok{(}\FloatTok{0.95}\NormalTok{,}\FloatTok{0.99}\NormalTok{,}\FloatTok{0.8}\NormalTok{)}
\NormalTok{test.results }\OtherTok{\textless{}{-}} \FunctionTok{simulation.3.test}\NormalTok{(prev,n.tested,Se,Sp)}

\FunctionTok{runjags.options}\NormalTok{(}\AttributeTok{silent.jags=}\ConstantTok{TRUE}\NormalTok{)}
\NormalTok{n.burnin }\OtherTok{\textless{}{-}}\NormalTok{ n.sample }\OtherTok{\textless{}{-}} \DecValTok{5000}
\NormalTok{results.jags }\OtherTok{\textless{}{-}} \FunctionTok{run.jags}\NormalTok{(}\StringTok{\textquotesingle{}three.test.jags\textquotesingle{}}\NormalTok{, }\AttributeTok{n.chains=}\DecValTok{2}\NormalTok{, }\AttributeTok{burnin=}\NormalTok{n.burnin, }\AttributeTok{sample=}\NormalTok{n.sample)}
\end{Highlighting}
\end{Shaded}

\begin{verbatim}
## Loading required namespace: rjags
\end{verbatim}

\begin{verbatim}
## Warning: No initial values were provided - JAGS will use the same initial
## values for all chains
\end{verbatim}

\begin{verbatim}
## Finished running the simulation
\end{verbatim}

\begin{Shaded}
\begin{Highlighting}[]
\FunctionTok{pander}\NormalTok{(}\FunctionTok{summary}\NormalTok{(results.jags))}
\end{Highlighting}
\end{Shaded}

\begin{longtable}[]{@{}
  >{\centering\arraybackslash}p{(\columnwidth - 12\tabcolsep) * \real{0.1667}}
  >{\centering\arraybackslash}p{(\columnwidth - 12\tabcolsep) * \real{0.1528}}
  >{\centering\arraybackslash}p{(\columnwidth - 12\tabcolsep) * \real{0.1250}}
  >{\centering\arraybackslash}p{(\columnwidth - 12\tabcolsep) * \real{0.1389}}
  >{\centering\arraybackslash}p{(\columnwidth - 12\tabcolsep) * \real{0.1250}}
  >{\centering\arraybackslash}p{(\columnwidth - 12\tabcolsep) * \real{0.1389}}
  >{\centering\arraybackslash}p{(\columnwidth - 12\tabcolsep) * \real{0.1389}}@{}}
\caption{Table continues below}\tabularnewline
\toprule()
\begin{minipage}[b]{\linewidth}\centering
~
\end{minipage} & \begin{minipage}[b]{\linewidth}\centering
Lower95
\end{minipage} & \begin{minipage}[b]{\linewidth}\centering
Median
\end{minipage} & \begin{minipage}[b]{\linewidth}\centering
Upper95
\end{minipage} & \begin{minipage}[b]{\linewidth}\centering
Mean
\end{minipage} & \begin{minipage}[b]{\linewidth}\centering
SD
\end{minipage} & \begin{minipage}[b]{\linewidth}\centering
Mode
\end{minipage} \\
\midrule()
\endfirsthead
\toprule()
\begin{minipage}[b]{\linewidth}\centering
~
\end{minipage} & \begin{minipage}[b]{\linewidth}\centering
Lower95
\end{minipage} & \begin{minipage}[b]{\linewidth}\centering
Median
\end{minipage} & \begin{minipage}[b]{\linewidth}\centering
Upper95
\end{minipage} & \begin{minipage}[b]{\linewidth}\centering
Mean
\end{minipage} & \begin{minipage}[b]{\linewidth}\centering
SD
\end{minipage} & \begin{minipage}[b]{\linewidth}\centering
Mode
\end{minipage} \\
\midrule()
\endhead
\textbf{prev} & 0.4634 & 0.5003 & 0.5347 & 0.4999 & 0.01853 & 0.5002 \\
\textbf{Se{[}1{]}} & 0.03515 & 0.4193 & 0.8257 & 0.4255 & 0.3694 &
0.05654 \\
\textbf{Se{[}2{]}} & 1.97e-05 & 0.3836 & 0.8115 & 0.3988 & 0.3851 &
0.01406 \\
\textbf{Se{[}3{]}} & 0.1572 & 0.5821 & 0.9714 & 0.5689 & 0.3818 &
0.9503 \\
\textbf{Sp{[}1{]}} & 0.1731 & 0.5857 & 0.9647 & 0.5746 & 0.3698 &
0.9439 \\
\textbf{Sp{[}2{]}} & 0.1879 & 0.6207 & 1 & 0.6013 & 0.3855 & 0.9864 \\
\textbf{Sp{[}3{]}} & 0.02819 & 0.4142 & 0.8429 & 0.431 & 0.3825 &
0.04896 \\
\bottomrule()
\end{longtable}

\begin{longtable}[]{@{}
  >{\centering\arraybackslash}p{(\columnwidth - 10\tabcolsep) * \real{0.1667}}
  >{\centering\arraybackslash}p{(\columnwidth - 10\tabcolsep) * \real{0.1667}}
  >{\centering\arraybackslash}p{(\columnwidth - 10\tabcolsep) * \real{0.1389}}
  >{\centering\arraybackslash}p{(\columnwidth - 10\tabcolsep) * \real{0.1111}}
  >{\centering\arraybackslash}p{(\columnwidth - 10\tabcolsep) * \real{0.1667}}
  >{\centering\arraybackslash}p{(\columnwidth - 10\tabcolsep) * \real{0.1111}}@{}}
\toprule()
\begin{minipage}[b]{\linewidth}\centering
~
\end{minipage} & \begin{minipage}[b]{\linewidth}\centering
MCerr
\end{minipage} & \begin{minipage}[b]{\linewidth}\centering
MC\%ofSD
\end{minipage} & \begin{minipage}[b]{\linewidth}\centering
SSeff
\end{minipage} & \begin{minipage}[b]{\linewidth}\centering
AC.10
\end{minipage} & \begin{minipage}[b]{\linewidth}\centering
psrf
\end{minipage} \\
\midrule()
\endhead
\textbf{prev} & 0.0002888 & 1.6 & 4115 & -0.006581 & 1.01 \\
\textbf{Se{[}1{]}} & 0.005599 & 1.5 & 4352 & 0.004059 & 25.76 \\
\textbf{Se{[}2{]}} & 0.006141 & 1.6 & 3933 & 0.03161 & 12.34 \\
\textbf{Se{[}3{]}} & 0.005797 & 1.5 & 4339 & -0.01245 & 25.79 \\
\textbf{Sp{[}1{]}} & 0.005542 & 1.5 & 4451 & -0.009849 & 51.99 \\
\textbf{Sp{[}2{]}} & 0.006308 & 1.6 & 3734 & 0.005016 & 55.55 \\
\textbf{Sp{[}3{]}} & 0.00551 & 1.4 & 4818 & -0.003747 & 25.84 \\
\bottomrule()
\end{longtable}
\end{frame}

\begin{frame}{How good is our posterior?}
\protect\hypertarget{how-good-is-our-posterior}{}
\begin{itemize}
\tightlist
\item
  Eyeball posterior distribution
\item
  Are the means (medians, modes?) close to the values used for the
  simulation
\item
  If we wish to be more formal about this we would repeat teh
  simulation-analysis cycle many (400+) times

  \begin{itemize}
  \tightlist
  \item
    this takes a long time, typically
  \item
    which is a better (less biased) predictor: mean, median or mode
  \item
    are the 95\% Credible Intervals true 95\% Confidence Intervals
  \end{itemize}
\end{itemize}
\end{frame}

\begin{frame}{Posterior predicted p-values}
\protect\hypertarget{posterior-predicted-p-values}{}
\begin{itemize}
\tightlist
\item
  But we're Bayesian aren't we?
\item
  Surely p-values belong to the frequentists!
\item
  Sometimes it is useful to compare data to a null hypothesis
\end{itemize}
\end{frame}

\begin{frame}{Calculating a posterior predicted p-value}
\protect\hypertarget{calculating-a-posterior-predicted-p-value}{}
\begin{itemize}
\tightlist
\item
  Simulate data based on the posterior distribution of the parameters
\item
  Typically 1 data set per iteration, using all parameters of interest
\item
  Define a metric/statistic to use
\item
  Calculate the metric for both the original data and for each simulated
  data set
\item
  The distribution of the metric from the simulated data sets represents
  its distribution under the null hypothesis that the model, and the
  posterior distributions are correct
\item
  Hence how extreme are the data under this null hypothesis
\item
  Low or high p-values (close to 0 or 1) give us cause for concern that
  the model is wrong.
\item
  What metric should we use?
\end{itemize}
\end{frame}

\begin{frame}{Quick exercise}
\protect\hypertarget{quick-exercise}{}
\begin{itemize}
\tightlist
\item
  Simulate some data \textasciitilde N(0,1)
\item
  Use JAGS to estimate the posterior for the mean and precision
\item
  Use the mean and precision to generate some more data
\item
  \textbf{Calculate} the mean and variance of the samples
\item
  Where do the mean and variance of the data sit wrt the samples?
\item
  What happens if we mis-specify the priors?
\end{itemize}
\end{frame}

\begin{frame}[fragile]{My version}
\protect\hypertarget{my-version}{}
\begin{Shaded}
\begin{Highlighting}[]
\NormalTok{n.obs }\OtherTok{\textless{}{-}} \DecValTok{100}
\NormalTok{obs }\OtherTok{\textless{}{-}} \FunctionTok{rnorm}\NormalTok{(n.obs,}\DecValTok{0}\NormalTok{,}\DecValTok{1}\NormalTok{)}
\NormalTok{obs.mu }\OtherTok{\textless{}{-}} \FunctionTok{mean}\NormalTok{(obs)}

\FunctionTok{cat}\NormalTok{(}
\StringTok{"model\{}
\StringTok{  \# Likelihood part:}
\StringTok{  for(i in 1:n.obs) \{}
\StringTok{    obs[i] \textasciitilde{} dnorm(mu, tau)}
\StringTok{  \}}
\StringTok{    }
\StringTok{  \# Prior part:}
\StringTok{  mu \textasciitilde{} dnorm(prior.mu.mu, prior.mu.tau)}
\StringTok{  tau \textasciitilde{}dgamma(0.001,0.001)}
\StringTok{  }
\StringTok{  \# Simulation part}
\StringTok{  for(i in 1:n.obs) \{}
\StringTok{    sim.obs[i] \textasciitilde{}dnorm(mu, tau)}
\StringTok{  \}}
\StringTok{  sim.mu \textless{}{-} mean(sim.obs)}
\StringTok{  obs.mean.higher \textless{}{-} obs.mu \textgreater{} sim.mu}
\StringTok{  }
\StringTok{  \# Hooks for automatic integration with R:}
\StringTok{  \#data\# obs, n.obs, obs.mu, prior.mu.mu, prior.mu.tau}
\StringTok{  \#monitor\# obs.mean.higher}
\StringTok{\}}
\StringTok{"}\NormalTok{, }\AttributeTok{file =} \StringTok{"ppp.mu.jags"}\NormalTok{)}

\NormalTok{prior.mu.mu }\OtherTok{\textless{}{-}} \DecValTok{0}
\NormalTok{prior.mu.tau }\OtherTok{\textless{}{-}} \FloatTok{0.001}
\FunctionTok{runjags.options}\NormalTok{(}\AttributeTok{silent.jags=}\ConstantTok{TRUE}\NormalTok{)}
\NormalTok{n.burnin }\OtherTok{\textless{}{-}}\NormalTok{ n.sample }\OtherTok{\textless{}{-}} \DecValTok{5000}
\NormalTok{results.jags }\OtherTok{\textless{}{-}} \FunctionTok{run.jags}\NormalTok{(}\StringTok{\textquotesingle{}ppp.mu.jags\textquotesingle{}}\NormalTok{, }\AttributeTok{n.chains=}\DecValTok{2}\NormalTok{, }\AttributeTok{burnin=}\NormalTok{n.burnin, }\AttributeTok{sample=}\NormalTok{n.sample)}
\end{Highlighting}
\end{Shaded}

\begin{verbatim}
## Warning: No initial values were provided - JAGS will use the same initial
## values for all chains
\end{verbatim}

\begin{verbatim}
## Finished running the simulation
\end{verbatim}

\begin{Shaded}
\begin{Highlighting}[]
\FunctionTok{pander}\NormalTok{(}\FunctionTok{summary}\NormalTok{(results.jags))}
\end{Highlighting}
\end{Shaded}

\begin{longtable}[]{@{}
  >{\centering\arraybackslash}p{(\columnwidth - 12\tabcolsep) * \real{0.2821}}
  >{\centering\arraybackslash}p{(\columnwidth - 12\tabcolsep) * \real{0.1282}}
  >{\centering\arraybackslash}p{(\columnwidth - 12\tabcolsep) * \real{0.1154}}
  >{\centering\arraybackslash}p{(\columnwidth - 12\tabcolsep) * \real{0.1282}}
  >{\centering\arraybackslash}p{(\columnwidth - 12\tabcolsep) * \real{0.1154}}
  >{\centering\arraybackslash}p{(\columnwidth - 12\tabcolsep) * \real{0.1154}}
  >{\centering\arraybackslash}p{(\columnwidth - 12\tabcolsep) * \real{0.1154}}@{}}
\caption{Table continues below}\tabularnewline
\toprule()
\begin{minipage}[b]{\linewidth}\centering
~
\end{minipage} & \begin{minipage}[b]{\linewidth}\centering
Lower95
\end{minipage} & \begin{minipage}[b]{\linewidth}\centering
Median
\end{minipage} & \begin{minipage}[b]{\linewidth}\centering
Upper95
\end{minipage} & \begin{minipage}[b]{\linewidth}\centering
Mean
\end{minipage} & \begin{minipage}[b]{\linewidth}\centering
SD
\end{minipage} & \begin{minipage}[b]{\linewidth}\centering
Mode
\end{minipage} \\
\midrule()
\endfirsthead
\toprule()
\begin{minipage}[b]{\linewidth}\centering
~
\end{minipage} & \begin{minipage}[b]{\linewidth}\centering
Lower95
\end{minipage} & \begin{minipage}[b]{\linewidth}\centering
Median
\end{minipage} & \begin{minipage}[b]{\linewidth}\centering
Upper95
\end{minipage} & \begin{minipage}[b]{\linewidth}\centering
Mean
\end{minipage} & \begin{minipage}[b]{\linewidth}\centering
SD
\end{minipage} & \begin{minipage}[b]{\linewidth}\centering
Mode
\end{minipage} \\
\midrule()
\endhead
\textbf{obs.mean.higher} & 0 & 1 & 1 & 0.5094 & 0.4999 & 1 \\
\bottomrule()
\end{longtable}

\begin{longtable}[]{@{}
  >{\centering\arraybackslash}p{(\columnwidth - 10\tabcolsep) * \real{0.3056}}
  >{\centering\arraybackslash}p{(\columnwidth - 10\tabcolsep) * \real{0.1528}}
  >{\centering\arraybackslash}p{(\columnwidth - 10\tabcolsep) * \real{0.1389}}
  >{\centering\arraybackslash}p{(\columnwidth - 10\tabcolsep) * \real{0.1111}}
  >{\centering\arraybackslash}p{(\columnwidth - 10\tabcolsep) * \real{0.1667}}
  >{\centering\arraybackslash}p{(\columnwidth - 10\tabcolsep) * \real{0.1250}}@{}}
\toprule()
\begin{minipage}[b]{\linewidth}\centering
~
\end{minipage} & \begin{minipage}[b]{\linewidth}\centering
MCerr
\end{minipage} & \begin{minipage}[b]{\linewidth}\centering
MC\%ofSD
\end{minipage} & \begin{minipage}[b]{\linewidth}\centering
SSeff
\end{minipage} & \begin{minipage}[b]{\linewidth}\centering
AC.10
\end{minipage} & \begin{minipage}[b]{\linewidth}\centering
psrf
\end{minipage} \\
\midrule()
\endhead
\textbf{obs.mean.higher} & 0.004925 & 1 & 10304 & -0.009387 & 0.9999 \\
\bottomrule()
\end{longtable}
\end{frame}

\begin{frame}[fragile]{Mis-specifying the prior}
\protect\hypertarget{mis-specifying-the-prior}{}
\begin{Shaded}
\begin{Highlighting}[]
\NormalTok{prior.mu.mu }\OtherTok{\textless{}{-}} \DecValTok{10}
\NormalTok{prior.mu.tau }\OtherTok{\textless{}{-}} \DecValTok{4}
\NormalTok{results.jags }\OtherTok{\textless{}{-}} \FunctionTok{run.jags}\NormalTok{(}\StringTok{\textquotesingle{}ppp.mu.jags\textquotesingle{}}\NormalTok{, }\AttributeTok{n.chains=}\DecValTok{2}\NormalTok{, }\AttributeTok{burnin=}\NormalTok{n.burnin, }\AttributeTok{sample=}\NormalTok{n.sample)}
\end{Highlighting}
\end{Shaded}

\begin{verbatim}
## Warning: No initial values were provided - JAGS will use the same initial
## values for all chains
\end{verbatim}

\begin{verbatim}
## Finished running the simulation
\end{verbatim}

\begin{Shaded}
\begin{Highlighting}[]
\FunctionTok{pander}\NormalTok{(}\FunctionTok{summary}\NormalTok{(results.jags))}
\end{Highlighting}
\end{Shaded}

\begin{longtable}[]{@{}
  >{\centering\arraybackslash}p{(\columnwidth - 12\tabcolsep) * \real{0.2895}}
  >{\centering\arraybackslash}p{(\columnwidth - 12\tabcolsep) * \real{0.1316}}
  >{\centering\arraybackslash}p{(\columnwidth - 12\tabcolsep) * \real{0.1184}}
  >{\centering\arraybackslash}p{(\columnwidth - 12\tabcolsep) * \real{0.1316}}
  >{\centering\arraybackslash}p{(\columnwidth - 12\tabcolsep) * \real{0.1053}}
  >{\centering\arraybackslash}p{(\columnwidth - 12\tabcolsep) * \real{0.1316}}
  >{\centering\arraybackslash}p{(\columnwidth - 12\tabcolsep) * \real{0.0921}}@{}}
\caption{Table continues below}\tabularnewline
\toprule()
\begin{minipage}[b]{\linewidth}\centering
~
\end{minipage} & \begin{minipage}[b]{\linewidth}\centering
Lower95
\end{minipage} & \begin{minipage}[b]{\linewidth}\centering
Median
\end{minipage} & \begin{minipage}[b]{\linewidth}\centering
Upper95
\end{minipage} & \begin{minipage}[b]{\linewidth}\centering
Mean
\end{minipage} & \begin{minipage}[b]{\linewidth}\centering
SD
\end{minipage} & \begin{minipage}[b]{\linewidth}\centering
Mode
\end{minipage} \\
\midrule()
\endfirsthead
\toprule()
\begin{minipage}[b]{\linewidth}\centering
~
\end{minipage} & \begin{minipage}[b]{\linewidth}\centering
Lower95
\end{minipage} & \begin{minipage}[b]{\linewidth}\centering
Median
\end{minipage} & \begin{minipage}[b]{\linewidth}\centering
Upper95
\end{minipage} & \begin{minipage}[b]{\linewidth}\centering
Mean
\end{minipage} & \begin{minipage}[b]{\linewidth}\centering
SD
\end{minipage} & \begin{minipage}[b]{\linewidth}\centering
Mode
\end{minipage} \\
\midrule()
\endhead
\textbf{obs.mean.higher} & 0 & 0 & 0 & 0.001 & 0.03161 & 0 \\
\bottomrule()
\end{longtable}

\begin{longtable}[]{@{}
  >{\centering\arraybackslash}p{(\columnwidth - 10\tabcolsep) * \real{0.3056}}
  >{\centering\arraybackslash}p{(\columnwidth - 10\tabcolsep) * \real{0.1667}}
  >{\centering\arraybackslash}p{(\columnwidth - 10\tabcolsep) * \real{0.1389}}
  >{\centering\arraybackslash}p{(\columnwidth - 10\tabcolsep) * \real{0.1111}}
  >{\centering\arraybackslash}p{(\columnwidth - 10\tabcolsep) * \real{0.1667}}
  >{\centering\arraybackslash}p{(\columnwidth - 10\tabcolsep) * \real{0.1111}}@{}}
\toprule()
\begin{minipage}[b]{\linewidth}\centering
~
\end{minipage} & \begin{minipage}[b]{\linewidth}\centering
MCerr
\end{minipage} & \begin{minipage}[b]{\linewidth}\centering
MC\%ofSD
\end{minipage} & \begin{minipage}[b]{\linewidth}\centering
SSeff
\end{minipage} & \begin{minipage}[b]{\linewidth}\centering
AC.10
\end{minipage} & \begin{minipage}[b]{\linewidth}\centering
psrf
\end{minipage} \\
\midrule()
\endhead
\textbf{obs.mean.higher} & 0.0003161 & 1 & 10000 & -0.001003 & 1.071 \\
\bottomrule()
\end{longtable}
\end{frame}

\begin{frame}{Random effects formulation used in medicine}
\protect\hypertarget{random-effects-formulation-used-in-medicine}{}
The ``standard'' approach, as used in the introductory course, to
specifying the interaction between two or more tests, is to add an
ammount to the probability that the two tests agree, and subtract the
same amount from the probability that they disagree:

\begin{longtable}[]{@{}lll@{}}
\toprule()
Animal + & Test 1 + & Test 1 - \\
\midrule()
\endhead
Test2 + & \(Se_1Se_2 + a\) & \((1-Se_1)Se_2 - a\) \\
Test2 - & \(Se_1(1-Se_2) - a\) & \((1-Se_1)(1-Se_2) + a\) \\
\bottomrule()
\end{longtable}
\end{frame}

\begin{frame}{The problems with this approach}
\protect\hypertarget{the-problems-with-this-approach}{}
\begin{itemize}
\tightlist
\item
  The possible range of a depends on \(Se_1\) and \(Se_2\) since these
  must all be probabilities and add up to 1
\item
  It is difficult to define the distribution of a
\end{itemize}
\end{frame}

\begin{frame}{Alternative formulations}
\protect\hypertarget{alternative-formulations}{}
\begin{itemize}
\tightlist
\item
  Define two latent states

  \begin{itemize}
  \tightlist
  \item
    e.g.~infection and seroconversion
  \item
    Matt tells me that this converges poorly
  \end{itemize}
\item
  ``Random effects'' model

  \begin{itemize}
  \tightlist
  \item
    ``standard'' approach, above can be considered a ``fixed effects''
    model
  \item
    ``random effect'' refers to a latent continuous variable associated
    with each individual
  \item
    the latent variable can be thought of as the propensity to test
    positive
  \item
    latent variables are relevant for both true positive animals (Se)
    and true negative animals (Sp)
  \item
    we need a monotonically increasing function limited by {[}0,1{]} to
    link the propensity of an individual and the probability that it
    tests positive
  \item
    a logistic model seems appropriate
    i.e.~\(\mathrm{logit}(P(T_{it}) = 1) = a_{td} + b_{d}*p_i, D_i=d\)

    \begin{itemize}
    \tightlist
    \item
      where \(P(T_{it} = 1)\) is the probability that individual i tests
      positive to test t
    \item
      \(a_{td}\) is a parameter to estimate for each test, t and disease
      status, d (diseased or not) and can be thought of as an estimate
      of teh test sensitivity or specificity
    \item
      \(p_i\) is the propensity for individual i to test positive, and
      is \(\sim{N(0,1)}\)
    \item
      \(b_d\) is a scaling factor since we use a standardised normal
      distribution for \(p_i\)
    \item
      an alternative formulation is to replace \(b_{d}*p_i\) with
      \(p_i \simN(0, Tau_d\)
    \end{itemize}
  \item
    to estimate the overall Se or Sp we need to integrate out \(p_i\)
  \item
    if \(b_d\) is zero (or all \(p_i\) are identical) then the
    Sensitivities/Specificities are conditionally independent
  \end{itemize}
\end{itemize}

\begin{block}{Thought exercise}
\protect\hypertarget{thought-exercise}{}
How would you:

\begin{enumerate}
[a)]
\tightlist
\item
  Simulate data to check to see if this works?
\item
  Code this in JAGS?
\end{enumerate}
\end{block}
\end{frame}

\end{document}
